% !Mode:: "TeX:UTF-8"
%!TEX program  = xelatex

\documentclass{cumcmthesis}
%\documentclass[withoutpreface,bwprint]{cumcmthesis} %去掉封面与编号页
\usepackage{float}
\usepackage{url}
\usepackage[framemethod=TikZ]{mdframed}
\title{}
\tihao{}
\baominghao{ 202017241008 }
\schoolname{华中科技大学}
\membera{李欣航}
\memberb{鲁镇仪}
\memberc{蒋瀚锐}
\supervisor{贺云峰}
\yearinput{2020}
\monthinput{09}
\dayinput{11}

\begin{document}
\maketitle
\begin{abstract}
	\par
	\textbf{对于问题一:}
	\par
	\textbf{对于问题二:}
	\par
	\textbf{对于问题三:}	\par
	\textbf{对于问题四:}	\par
	\keywords{元胞自动机 \quad  \quad  }
\end{abstract}

%\tableofcontents
\section{问题重述}
\subsection{问题背景}
% \begin{figure}[!h]
% 	\centering
% 	\includegraphics[width=0.8\textwidth]{Figure_1.pdf}
% 	\caption{模型示意图}
% \end{figure}

\subsection{问题的提出}
该文章主要是研究
\par

\par
在问题一当中,
\par
在问题二当中,
\par
在问题三当中,
\par
在问题四当中,
\newpage
\section{问题分析}
\subsection{问题一的分析}
问题一较为简单,\par

\subsection{问题二的分析}
问题二
\subsection{问题三的分析}

在问题三当中,
\subsection{问题四的分析}
问题四中

\newpage
\section{模型假设}
\begin{itemize}
	\item 认为
	\item 假设
\end{itemize}

\newpage
\section{符号说明}
\begin{table}[H]
	\caption{符号说明}
	\centering
	\begin{tabular}{ccc}
		\toprule[1.5pt]
		符号 & 含义  & 单位    \\
		\midrule[1pt]
		$x$  & $x$轴 & 元/$km$ \\
		\bottomrule[1.5pt]
	\end{tabular}
\end{table}


\newpage
\section{模型总结}
\subsection{模型改进}

\subsection{模型的优缺点}

\textbf{模型的优点:}
\begin{itemize}
	\item[1] 问题模型
	\item[2] 问题一模型优点:
	\item[3] 问题三模型优点:
	\item[4] 问题四模型优点:
\end{itemize}

\textbf{模型的缺点:}
\begin{itemize}
	\item[1] 问题一模型缺点:
	\item[2] 问题三模型缺点:
	\item[3] 问题四模型缺点:
\end{itemize}
\newpage
%参考文献
\bibliographystyle{unsrt}%按引用的先后顺序
\bibliography{mainpaper}

\newpage
%附录
\begin{appendices}
	\section{软件版本}
	\center{MATLAB R2020a}
	\center{Python 3.8}

	\section{源程序}
	\subsection*{C++ 源代码}
	\begin{lstlisting}[language=c++]
int main(){
	return 0;
}
\end{lstlisting}

\end{appendices}

\end{document}